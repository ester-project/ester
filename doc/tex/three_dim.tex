\chapter{Stars in three dimensions}
\label{chap:3D}


\section{Introduction}

The need to deal with three-dimensional stellar models comes from the common
situation where the large scales of a star do not own any symmetry. Typically
that occurs when the star owns a large-scale magnetic field with no special
symmetry or if the star is in the gravitational field of another massive object.
In the general case the situation is even more complex since the star is
usually not in a steady state: a variable tidal potential is a common case.
However, the first step to deal with these situations is to focus on a steady
state that may only exist as a time-averaged state. Fortunately, in a
steady state we can build on the previous 2D approach and devise the
tools for constructing the 3D stellar models.

The first step is to derive the 3D mapping, which maps the spheroidal shape of
the star to the spherical coordinates. We are still inspired by the work of
\cite{BGM98}, who also considered the question of 3D stellar models to deal with
binary neutron stars. Their work however aimed at including the General
Relativity formalism to include strong gravitational field effects
\citep{BGM99}.

\section{The mapping}

We follow the 2D case described in chap. \ref{chap:mapping2D} but with
a spheroid described by

\begin{equation}
\left\{
\begin{array}{l}
r=r(\zeta,\theta',\varphi')\\
\theta=\theta'\\
\varphi=\varphi'
\end{array}
\right.
\label{themap3D}
\end{equation}
The angular variables remain the spherical ones but the radial distance now
depends on $(\theta',\varphi')$.

\subsection{Question of symmetry}

In chapter \ref{chap:mapping2D} we implicitly assumed that the star was
symmetric with respect to equator, the consequence of which  being that
the surface verifies $R(\theta)=R(\pi-\theta)$ and $A_0(\zeta)$ is a
polynomial of odd order in $\zeta$. We now discuss this choice since we
need to consider stars of any shape as long as they are topologically
equivalent to a sphere.

Let us consider a stellar surface of equation

\beq r=R(\theta,\varphi)\eeq
We can always split $R(\theta,\varphi)$ into its symmetric and
anti-symmetric parts with respect to origin, namely in the
transformation

\[ (\theta,\varphi) \tv (\pi-\theta,\varphi+\pi)\]
Hence, we write

\beq R(\theta,\varphi) = R_s(\theta,\varphi) + \delta R_a(\theta,\varphi)
\eeq
where

\greq
R_s(\theta,\varphi)=R_s(\pi-\theta,\varphi+\pi) \\
\delta R_a(\theta,\varphi) = -\delta R_a(\pi-\theta,\varphi+\pi)
\egreq
Our notation underlines the fact that deviations from central symmetry
are usually small.

The splitting of $R(\theta,\varphi)$ into its symmetric and
anti-symmetric parts implies some constraints on the mapping
$r(\zeta,\theta',\varphi')$. Indeed, we require that $r\sim\zeta$ near
the centre, namely that the new coordinate system behaves just like the
spherical one. Basically, we must have

\beq \vr(r(-\zeta,\theta,\varphi),\theta,\varphi) =
\vr(r(\zeta,\pi-\theta,\varphi+\pi),\pi-\theta,\varphi+\pi) \eeq
which implies

\beq r(-\zeta,\theta,\varphi) = -r(\zeta,\pi-\theta,\varphi+\pi) \eeqn{sym_rel}

\subsubsection{The axisymmetric case}

Let's come back to the axisymmetric where we may write

\beq r(\zeta,\theta,\varphi) = a\zeta + A_o(\zeta)(R_s(\theta)-a)+
A_e(\zeta)\delta R_a(\theta)\with 0\leq\zeta\leq1 \eeqn{map_axi}
Symmetry constraint \eq{sym_rel} imposes the parity of the $A_o$ and $A_e$
functions, namely that

\beq A_o(-\zeta) = -A_o(\zeta) \andet A_e(-\zeta) = A_e(\zeta) \eeq
meaning that $A_o(\zeta)$ is a polynomial of odd order while $A_e(\zeta)$
is of even order.

Hence, for the lowest order polynomials that verify $A(0)=0$, $A(1)=1$ and
$A'(0)=0$, $A'(1)=0$, we can choose

\beq A_o(\zeta)=\demi(5\zeta^3-3\zeta^5) \andet
A_e(\zeta)=2\zeta^2-\zeta^4 \eeqn{poly_low}
We note that the expression of $A_o(\zeta)$ is the same as in
\eq{exp_a0} and agrees with that of \cite{BGM98}. The expression of
$A_e(\zeta)$ differs from that of \cite{BGM98} who prefer $A_e(\zeta) =
3\zeta^4-2\zeta^6$ thus of higher order. The advantage of $A_e(\zeta)$
expression in \eq{poly_low} is that it induces less coupling between the
Chebyshev polynomials describing the solution. However, the polynomial
proposed by \cite{BGM98} vanishes more rapidly near the origin thus
inducing less coupling in the spherical harmonics expansion of the
solution. The mapping is closer to the spherical coordinates near the
centre. The best of the two possibilities is not known and has to be
determined by numerical experiments.

\subsubsection{The non-axisymmetric case}

We can now just generalize \eq{map_axi} to the non-axisymmetric case, namely

\beq r(\zeta,\theta,\varphi) = a\zeta + A_o(\zeta)(R_s(\theta,\varphi)-a)+
A_e(\zeta)\delta R_a(\theta,\varphi)\eeqn{map_naxi}
Since we'll be using spectral method, functions of $(\theta,\varphi)$
are expanded on the spherical harmonics. We recall that

\beq \YL(\pi-\theta,\varphi+\pi) = (-1)^\ell\YL(\theta,\varphi) \eeq
which implies that

\beq R_s(\theta,\varphi) = \sum_{\ell=0,2,\ldots} \sum_{m=-\ell}^{+\ell} a_m^\ell\YL(\theta,\varphi) \eeq
\beq \delta R_a(\theta,\varphi) = \sum_{\ell=1,3,\ldots} \sum_{m=-\ell}^{+\ell} b_m^\ell\YL(\theta,\varphi) \eeq
If we assume that the star is in equilibrium in a steady tidal potential
then we can set $\delta R_a=0$ and can restrict expansions to even $m$'s.
This is the first step to do.




