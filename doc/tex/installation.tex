\chapter{Getting started}
\section{Prerequisites}

The ESTER libraries depend on some external libraries that should be installed in
the system, namely:
\begin{itemize}
    \item BLAS, CBLAS and LAPACK, for matrix algebra. There are several
        alternatives available, as for example:
        \begin{itemize}
            \item Netlib. This is the original implementation. The LAPACK
                library can be found at \url{http://www.netlib.org/lapack}, and
                already contains BLAS, but CBLAS should be downloaded separately
                from \url{http://www.netlib.org/blas}.
            \item ATLAS (Automatically Tuned Linear Algebra Software). An
                implementation of LAPACK/BLAS that is automatically optimized
                during the compilation process. It can be found at
                \url{http://math-atlas.sourceforge.net/}. It contains LAPACK,
                BLAS and CBLAS.
            \item Intel MKL. Contains an optimized version of LAPACK, BLAS and
                CBLAS for Intel processors.
        \end{itemize}
    \item PGPLOT (CPGPLOT) for graphics output (optional). PGPLOT is available
        at \url{http://sourceforge.net/projects/pg2plplot/} or, in most Linux
        distributions, in the \texttt{pgplot5} package.
    \item HDF5 for standardized model output (optional). HDF5 is available at
        \url{http://www.hdfgroup.org/downloads/}.
\end{itemize}
% As there are some routines written in Fortran, it is also needed to link against
% the standard fortran libraries ({\tt libgfortran} for the GNU fortran compiler
% and {\tt libifcore} and {\tt libifport} for the Intel compiler).

\subsection{A note about the performance of the code}
The performance of the ESTER code depends strongly on LAPACK. To get the best results,
use an optimized (and parallelized) version.

\pagebreak

\section{Installation}

The latest version of ESTER is available from the \git repository:
\begin{shell}
    $ git clone https://github.com/ester-project/ester.git
    !$
\end{shell}
or from a source tarball: \url{http://ester-project.github.io/ester/}.

the tarball source is for users who need a stable version and develop
their own applications from it. The git repository version is for users
who want to keep up with the lastest version of the code or even want to
contribute to its development.

If you choose to get ESTER from the \git repository, you will need to have
\texttt{libtool}, \texttt{autoconf} and \texttt{automake} installed.
The first step after cloning the repository is to run the \texttt{bootstrap}
script:
\begin{shell}
    $ cd ester
    ester$ ./bootstrap
\end{shell}
This will create the \texttt{configure} script.

\subsection{Configure}
In this step, the \texttt{configure} script will detect the libraries and
compiler installed on the system.
It us preferred to configure and compile in a different directory than the top
source directory:
\begin{shell}
    ester$ mkdir BUILD
    ester$ cd BUILD
    ester/BUILD$ ../configure [OPTION]... [VAR=VALUE]...
    !$
\end{shell}

The most important configure options are:
\begin{description}
    \item[--prefix:] selects the installation directory for binaries,
ester libraries, etc. (default is \texttt{\$HOME/local}.
    \item[--enable-hdf5:] to enable HDF5 support (this requires to have HDF5
        library installed on the system).
    \item[--help:] prints help and the full list of configure options.
\end{description}
The following variables can be used to tune the ESTER build configuration:
\begin{description}
    \item[FC:] Fortran compiler to be used (\eg FC=ifort)
    \item[CC:] C compiler
    \item[CXX:] C++ compiler
\end{description}

For instance, is you want to compile with Intel compilers and install ESTER in
\texttt{\$HOME/ester\_local}, you should use:
\begin{shell}
    ester/BUILD$ ../configure --prefix=$HOME/ester\_local CC=icc CXX=icpc FC=ifort
\end{shell}


\subsection{Build and Install}
After the configure step, building and installing ESTER is straightforward:
\begin{shell}
    ester/BUILD$ make && make install
    !$
\end{shell}
\texttt{make} will build the ESTER's libraries and binaries.
And \texttt{make install} will copy the libraries into \texttt{\$prefix/lib} and
binaries into \texttt{\$prefix/bin}.

Make sure you add the install directory to you \texttt{PATH} environment
variable to be able to launch ester without specifying the full path to the
binary.
If you are using \texttt{bash}, you can add the following line to your
\texttt{.bashrc}:
\begin{shell}
    export PATH="$HOME/local/bin:$PATH"
\end{shell}


\subsection{Updating the code}
If you chose to download ESTER from the \git repository, you can update the code with:
\begin{shell}
    ester$ git pull
    !$
\end{shell}
and compile the new version by going to your build directory and running
\texttt{make install}:
\begin{shell}
    ester$ cd BUILD
    ester/BUILD$ make install
\end{shell}

\section{Checking the Installation}

To check the installation, a series of test runs is proposed: following
the previous example, in the directory \tt{\$HOME/ester\_local/bin},
where binaries have been built, just run {\tt ester-check-models}. These
tests first compute a 1D 5~\msun model both with native and hdf5 output
and compare the results to a reference output. A 2D 5~\msun model with
$\omega=0.5\omega_k$ is then computed and checked again for the two
kinds of output. The same check is repeated for a 10\msun model but with
$\omega=0.3\omega_k$.



%To check the functionality of the program we are going to calculate the
%structure of a star using the default values for the parameters.
%
%First create and go to directory where to save models:
%\begin{shell}
%    $ mkdir /tmp/models
%    $ cd /tmp/models
%\end{shell}
%
%Then, we calculate the structure of the corresponding 1D non-rotating star:
%\begin{shell}
%    /tmp/models$ ester 1d
%    !$
%\end{shell}
%After this step, the directory should contain file named \texttt{star.out} this
%is the model of the 1D non-rotating star.
%We can use this model as the starting point for the calculation of the 2D
%rotation model:
%\begin{shell}
%    /tmp/models$ ester 2d -i star.out -Omega_bk 0.5
%    !$
%\end{shell}
%The \texttt{-i} option specify the input model, and \texttt{-Omega\_bk} gives
%the rotation velocity as a fraction of the break-up velocity.
%In this example the star is rotating at 50\% of the break-up velocity
%($\Omega_k=\sqrt{\frac{GM}{R_e^3}}$).


% The current version of the ESTER code can be downloaded using {\tt svn} from the project server by doing
% \mint{bash}|$ svn checkout http://ester-project.googlecode.com/svn/trunk/ ester| %$
% or from the project website \url{http://code.google.com/p/ester-project}.
% 
% The first step is to create the file {\tt make.inc} in the directory {\tt src}
%  from the two examples that are included, 
% {\tt make.inc.icc} and {\tt make.inc.gcc}, for the Intel compiler and the GNU compiler respectively.
% After setting the appropriate values for the compilation, we must start by doing 
% \mint{bash}|ester/src$ make tables| %$
% This will build some third-party libraries included in the distribution and initialise
% the tables of opacity and equation of state.
% 
% We can now build the main program by doing
% \mint{bash}|ester/src$ make| %$
% To remove intermediate files we can also do 
% \mint{bash}|ester/src$ make clean| %$
% Finally, to verify the installation, we can do
% \mint{bash}|ester/src$ make test| %$
% 
% The main executable is located in {\tt ester/bin/}. To be able to call ester 
% without including the full path, you can include this directory in your PATH
% environment variable. Alternatively, you can create a symbolic link in
% a directory included in your PATH, for example:
% \mint{bash}|$ ln -s ~/ester/bin/ester ~/bin/ester| %$
% In this example, we are supposing that the ESTER library is located in \verb|~/ester|
% and the directory \verb|~/bin| is included in the PATH. If you are interested also
% in making your own programs using the ESTER library, you can also do:
% \mint{bash}|$ ln -s ~/ester/bin/ester_build ~/bin/ester_build| %$
% 
% \subsection{Updating the code}
% In order to update to the last version using {\tt svn}, from the root directory of the ESTER distribution
% execute
% \mint{bash}|ester$ svn update| %$
% Depending on the update, sometimes we can do just
% \mint{bash}|ester/src$ make| %$
% from the {\tt src} directory. But it is safer to clean out the previous installation using
% \mint{bash}|ester/src$ make distclean| %$
% and then
% \mint{bash}|ester/src$ make tables; make| %$
% 
% \section{Checking the installation}
% 
% To check the functionality of the program we are going to calculate the structure of a star using the default values for the parameters.
% First we calculate the structure of the corresponding 1D non-rotating star. Change to your working directory and execute
% \mint{bash}|$ ester 1d| %$
% Then we use the output file (by default {\tt star.out}) as the starting point for the 2D calculation
% \mint{bash}|$ ester 2d -i star.out -Omega_bk 0.5|   %$
% This calculates the structure of a star rotating at 50$\%$ of the break-up velocity $\Omega_k=\sqrt{\frac{GM}{R_e^3}}$.
% 
% \section{Using the library}
% 
% The ESTER code can be used as a C++ library. We just have to add the following line at the beginning of our
% C++ program
% \mint{bash}|#include "ester.h"|
% The main library is created in {\tt ester/lib/libester.so} and the header files
% are in {\tt ester/include}.
% To facilitate the process of compiling and linking against the library and all its dependencies, we provide an automatically generated
% script {\tt ester/bin/ester\_build} so, all you have to do is
% \mint{bash}|$ ester_build your_cpp_program.cpp -o your_executable| %$






